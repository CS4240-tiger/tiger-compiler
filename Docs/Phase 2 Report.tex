\documentclass[a4paper]{article}

\usepackage[english]{babel}
\usepackage[utf8]{inputenc}
\usepackage{amsmath}
\usepackage{graphicx}
\usepackage[colorinlistoftodos]{todonotes}
\usepackage{indentfirst}

\addto{\captionsenglish}{\renewcommand{\abstractname}{Run/Compile Instructions}}

\title{CS 4240 Project Phase 2}

\author{David Benas, Sean Collins, Casey Evanish}

\date{\today}

\begin{document}

\maketitle

\begin{abstract}
Instructions on how to run the code / compile and such will go here.
\end{abstract}

\section{Symbol Table}
	As a group, we decided to have variables, types, and functions as symbols in our table. Variables would relate to an initialization value, types would have a backing type (obviously), and functions would have a return type. At first, we wanted functions to have their params in the symbol table as well, but we realized that since functions cannot share names, this was unecessary. 

Ultimately, these three entries in the symbol table share the fact that they have a scope and an ID. We represented our Symbol Table as a Hashmap with a String key and a SymbolTableEntry(containing the parent scope and a String ID) value.

An example of how we add symbols into our table is as follows [from tiger.g]:
\begin{verbatim}
type[String id]	
  :	base_type {
    if ($base_type.text.equals("int")) { 
      symbolTable.put(new TypeSymbolTableEntry(CURRENT_SCOPE,strip(id), 
      				  TigerPrimitive.INT));
    } else if ($base_type.text.equals("fixedpt")) {
      symbolTable.put(new TypeSymbolTableEntry(CURRENT_SCOPE,strip(id), 
      				  TigerPrimitive.FIXEDPT));
    }
  }
\end{verbatim}

In order to deal with scoping, we first instantiate a scope object. Whenever we encounter a new block, we create a new scope object, while setting its parent to be the previous scope. The scope objects contain their parent scope, ID, and a counter for number of children. Essentially, we can think of our scopes as an upside-down tree, with each parent scope branching down to many children scopes. The way we actually use scoping can be seen in output\textbackslash SymbolTable.java 's put() method: \\ \\ \\ \\
\begin{verbatim}
for (int index = 0; index < resultVarList.size(); index++) {
     while (curScope != null) {
        if (curScope.equals(resultVarList.get(index))) {
        // Found value in this or parent Scope! Reassign it.
        resultVarList.get(index).setValue(addVar.getValue());
        return;
        }						
        curScope = curScope.getParent();
     }					
}
\end{verbatim}
Little bit more of scoping, explaining above quoteblock and how it applies to code goes here.
\section{Semantic Checking}

\section{Intermediate Code}



\end{document}